
\section{What Does the Present Tell Us About the Past?}

Planets are inextricably linked to their host stars through their common birth environment.  Numerous intertwined astrophysical processes, ranging from tidal forces to stellar high-energy emission, shape the long-term evolution of planetary systems. Modern astronomical surveys have produced a wealth of information to both discover planet-hosting stars and help elucidate the processes that govern their evolution. Notably, completed missions such as \kepler and \textit{K2}, and the in-progress Transiting Exoplanet Survey Satellite (TESS), monitor the long-term bright variation of thousands of stars to detect transiting exoplanets and eclipsing binary stars, massively increasing the known population of such systems.  Furthermore, future missions such as the James Webb Space Telescope (JWST) will attempt to detect and characterize the first terrestrial planetary atmospheres, likely the inner-most planets of the nearby TRAPPIST-1 system \citep{Lustig2019}. With such large amounts of data comes an unprecedented opportunity for both population-level statistical studies and individual target characterization to infer the fundamental properties of the populations of stars and their planets.

An unavoidable problem for astronomy, however, is that the timescales over which astrophysical processes operate are often much, much longer than the lifetime of a single astronomer, or even the duration of recorded human history. For example, the lifetime of Sun-like stars is of order 10 billion years, a factor of $10^7$ longer than an optimistic estimate for the lifetime of a person. How can we hope to learn about the long-term evolution of such objects? Astronomers are therefore reduced to observing brief snapshots of the cosmos over time, dramatically complicating the challenge of teasing out what processes govern what we observe. The fundamental problem of astronomy therefore becomes, given what we observe, how can we infer what processes produced what we see, and what does this information tell us about the past and future evolution of this celestial object? 

Interpreting and understanding observations of astrophysical systems requires a theoretical model.  In the case where observations have not yet been made, e.g. the detection of a terrestrial exoplanetary atmosphere, theoretical models can be used to make predictions that will aid in the interpretation of such observations.  For potentially habitable terrestrial exoplanets, for example, \citet{Luger2015} showed how large pre-main sequence luminosities of low-mass stars drive water photolysis during an extended runaway greenhouse phase, potentially causing extreme amounts of O$_2$ to build up in their atmospheres, significantly impacting the exoplanets' observed spectra.  When observations are available, model predictions can be conditioned on the observations to derive constraints on model parameters, that is, astronomers can attempt to match their model outputs with the observations within the uncertainties. Typically, model parameters represent physically meaningful quantities, like an exoplanet's radius or an atmospheric chemical abundance, that we hope to infer to understand the underlying astrophysical processes at work.

When future facilities, like JWST, go online and attempt to characterize terrestrial exoplanetary atmospheres in the search for biosignatures, astronomers will use complex radiative transfer codes to derive atmospheric compositions \citep[e.g. SMART;][]{Meadows1996,Crisp1997}. Critical to understanding and interpreting these results is determining how the exoplanet and its atmosphere evolved to its present state. To answer this question, one must use theoretical models, conditioned on the observations, to simulate the system's long-term evolution.  These problems are not limited to observations of exoplanets, however, as observations of stellar systems, e.g. \kepler eclipsing binary orbital and rotation synchronization measurements \citep{Lurie2017}, also require theoretical modeling of their long-term evolution to interpret observations.

Many theoretical studies examine the long-term evolution of systems to interpret observations and constrain parameters, i.e. exoplanetary rotation rates, stellar tidal Qs, etc, do so in the form of ``best fit models" by finding the set of model parameters that best reproduce the observations.  Another common technique is performing a grid search through parameter space to examine how model predictions vary as a function of model parameters.  For example, \citet{Ribas2016} and \citet{Barnes2016} both employed the aforementioned techniques when examining the long-term dynamical and atmospheric evolution of Proxima Centauri b to estimate its present-day potential habitability.  These techniques are useful and informative, but ultimately insufficient, however, as a robust analysis used to interpret observations must be based in a probabilistic framework in which model predictions are conditioned on the observations and the associated uncertainties, e.g. by using a statistical technique like Bayesian inference.  Any constrained parameter requires an uncertainty estimation to put the interpretation in context and permit a robust, credible interpretation of the results.  For example, if a forward model yields a best fit prediction of 1 Earth ocean of liquid surface water remaining on an exoplanet, uncertainties are required to interpret this result.  If instead, in this example, one propagated uncertainties through the forward model to find that the remaining surface water content is 1$^{+10}_{-1}$ Earth oceans, the planet's current state ranges from desiccated to water-rich, with both cases entirely consistent with the observations, yet indicative of significantly different evolutionary histories.  

\section{Dissertation Outline}

Below, I provide an outline for this dissertation and briefly highlight my results. In Chapter 2, I study the dynamics of the birth environment of circumbinary exoplanets by running an ensemble of smooth particle hydrodynamic (SPH) N-body simulations of circumbinary protoplanetary disks. My work demonstrates that these dynamically-rich disks coevolve with their central host binary stars, exchanging angular momentum through gravitational resonances, ultimately driving the orbital evolution of both the binary and the inner-edge of the disk where planets can form and migrate. In Chapter 3, I extend my work with circumbinary planetary systems and develop a model to explain the anomalous lack of observed circumbinary planets (CBPs) orbiting short-period binary stars in the \kepler field. By constructing a theoretical model for the coupled stellar-tidal evolution of binary stars, I show that the expanding orbits of young binaries can efficiently destabilize CBPs that preferentially orbit just exterior to the dynamical stability limit, explaining their observed dearth.

I continue my work with binary stars in Chapter 4 to explore how the competition between tidal torques and magnetic braking in binaries can shape the observed rotation period distribution of low-mass main sequence stars in the \kepler field. I show how my model reproduces previously-unexplained stellar populations in the \kepler field, such as the subsynchronous eclipsing binary stars identified by \citet{Lurie2017}. I apply my theory to identify major limitations of the stellar age determination method of gyrochronology. I then explore how my model's predictions could be used in tandem with future observations to distinguish between which theory best describes tidal interactions in main sequence low-mass binary stars.

In Chapter 5, I introduce \approxposterior, my open-source machine learning Python package for approximate Bayesian inference with computationally-expensive models. I discuss the algorithm, its convergence properties, and provide example use cases. I then discuss how this efficient machine learning method can enable Bayesian inference studies for a wide array of computationally-expensive inference problems. In Chapter 6, I combine theory, Bayesian inference, and machine learning to infer the long-term high-energy radiation environment experienced by the TRAPPIST-1 planetary system, the current best target for the detection and characterization of terrestrial planet atmospheres by the James Webb Space Telescope (JWST). I construct a probabilistic model for TRAPPIST-1's evolving XUV luminosity, conditioning the inference on both observations of TRAPPIST-1 and of late M-dwarfs. From this inference, I find that TRAPPIST-1 likely underwent an extended epoch of enhanced XUV emission, potentially driving extreme volatile loss from its planets, impacting what JWST might observe in future observations. I demonstrate that this Bayesian inference analysis is too computationally-expensive to scale to either a larger number of stars or inference with a more complex model. To address this concern, I applied \approxposterior to the TRAPPIST-1 inference problem. I demonstrate that it can accurately repeat the probabilistic analysis, but requires nearly three orders of magnitude less computational resources than \emcee.  Finally, I summarize my findings and discuss prospects for related future research.

