\setcounter{page}{-1}
\abstract{%

In this dissertation, I develop theoretical and probabilistic models to understand and infer the long-term evolution of single and binary stars. Although my modeling efforts focus on the dynamical evolution of such systems, e.g. how and why a binary's orbit evolves over time, I consider what impact stellar and binary evolutionary processes have on planets orbiting the stars. I employ my models to examine different aspects of the lifetime of stellar and planetary systems, including early stellar interactions with a protoplanetary disk and the evolving high-energy radiation environment experienced by terrestrial planets orbiting late M-dwarfs throughout the pre-main sequence. I confront my model predictions with observational data from missions like \kepler to constrain unobserved model parameters and develop testable hypotheses for unexplained features in the data. Through my novel theories and inference methods, I help determine how the past evolution of stars in planetary systems shape what we observe today.

To study the dynamics of the birth environment of exoplanets, I ran an ensemble of smooth particle hydrodynamic (SPH) N-body simulations of circumbinary protoplanetary disks. My work demonstrates that these dynamically-rich disks coevolve with their central host binary stars, exchanging angular momentum through gravitational resonances, ultimately driving the orbital evolution of both the binary and the inner-edge of the disk where planets can form and migrate. I extend my work with circumbinary planetary systems and develop a model to explain the anomalous lack of observed circumbinary planets (CBPs) orbiting short-period binary stars in the \kepler field. By constructing a theoretical model for the coupled stellar-tidal evolution of binary stars, I show that the expanding orbits of young binaries can efficiently destabilize CBPs that preferentially orbit just exterior to the dynamical stability limit, explaining their observed dearth.

I continue my work with binary stars to explore how the competition between tidal torques and magnetic braking in binaries can shape the observed rotation period distribution of low-mass main sequence stars in the \kepler field. I show how my model reproduces previously-unexplained stellar populations in the \kepler field, such as the subsynchronous eclipsing binary stars identified by \citet{Lurie2017}. I apply my theory to identify major limitations of the stellar age determination method of gyrochronology. I then explore how my model's predictions could be used in tandem with future observations to distinguish between which theory best describes tidal interactions in main sequence low-mass binary stars.

Finally, I use my models to infer the long-term high-energy radiation environment experienced by the TRAPPIST-1 planetary system, the current best target for the detection and characterization of terrestrial planet atmospheres by the James Webb Space Telescope (JWST). I construct a probabilistic model for TRAPPIST-1's evolving XUV luminosity, conditioning the inference on both observations of TRAPPIST-1 and of late M-dwarfs. From this inference, I find that TRAPPIST-1 likely underwent an extended epoch of enhanced XUV emission, potentially driving extreme volatile loss from its planets, impacting what JWST might observe in future observations. I demonstrate that this Bayesian inference analysis is too computationally-expensive to scale to either a larger number of stars or inference with a more complex model. To address this concern, I develop a novel open-source machine learning Python package, \approxposterior. I apply \approxposterior to the TRAPPIST-1 inference problem and demonstrate that it can accurately repeat the probabilistic analysis, but requires nearly three orders of magnitude less computational resources than \emcee. I conclude by discussing how my new efficient machine learning method can enable similar Bayesian inference studies for a wider array of inference problems to constrain the long-term evolution of stars and their planets.}
