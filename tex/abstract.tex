\setcounter{page}{-1}
\abstract{%

Modern surveys like NASA's \kepler mission have collected a wealth of data in the search for Earth-like exoplanets. This vast quantity of data has enabled novel statistical investigations of the physical processes that shape the observed populations of stars and their planets. Theoretical models are required to explain how and why planetary systems evolved to their present state because models produce hypotheses for how physical mechanisms operate that can be directly tested by observational data. One can therefore compare model predictions with observed data and its uncertainties, a process mathematically formalized by Bayesian inference, to infer and understand the long-term evolution of planetary systems. In this dissertation, I developed theoretical models to understand the long-term evolution of single and binary stars. My work focused on simulating the dynamical evolution of stellar systems and explored what impact this evolution had on the planetary system architecture and planetary habitability.  

Through an ensemble of N-body simulations, I demonstrated how resonant gravitational torques in young circumbinary systems can drive orbital eccentricity growth in both the central binary and its external circumbinary protoplanetary disk, depending on the binary orbital eccentricity. I continued my work on circumbinary systems and showed how the early tidally-driven expansion of short-period binary orbits can destabilize close-in circumbinary planets thereby explaining the lack of observed transiting circumbinary planets in the \kepler field. I then extended this model and applied it to \kepler binaries to probe how stellar evolution, tidal torques, and magnetic braking shape the rotation period evolution of low-mass binary stars. Using these models to infer the evolutionary history of stellar and exoplanetary systems is inherently computationally-expensive because it requires running a large number of simulations. To enable Bayesian inference with my models, I created an open-source Python machine learning package for efficient approximate Bayesian inference, \approxposterior. I applied my code to infer the X-ray and ultraviolet emission history of TRAPPIST-1 to understand the evolving high-energy radiation environment experienced by its planets.  In this dissertation, I combined theory, Bayesian inference, and machine learning to interpret observational data and characterize the long-term evolution of stars and their planets.}
