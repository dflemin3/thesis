\setcounter{page}{-1}
\abstract{%

Modern surveys like NASA's \kepler mission have collected a wealth of data in the search for Earth-like exoplanets. This vast quantity of data has enabled novel statistical investigations of the physical processes that shape the observed populations of stars and their planets. Theoretical models are required to explain how and why planetary systems evolved to their present state because models produce hypotheses for how physical mechanisms operate that can be directly tested by observational data. One can therefore compare model predictions with observed data and its uncertainties, a process mathematically formalized by Bayesian inference, to infer and understand the long-term evolution of planetary systems. In this dissertation, I developed theoretical models to understand the long-term evolution of single and binary stars. My work focused on simulating the dynamical evolution of stellar systems and explored what impact this evolution had on the planetary system architecture and planetary habitability.  

Through an ensemble of N-body simulations, I explored how resonant gravitational torques in young circumbinary systems impact the orbital evolution of the central binary and its external
circumbinary protoplanetary disk. I demonstrated that binaries with eccentric orbits strongly coupled to the disk and excited eccentricity growth for both the binary orbit and the disk. I found that binaries on nearly circular orbits, however, weakly coupled to the disk and only caused eccentricity growth within the disk. I continued my work on circumbinary systems to develop a model for the early coupled stellar-tidal evolution of planet-hosting binary stars. I showed how the early tidally-driven expansion of short-period binary orbits can destabilize close-in circumbinary planets thereby explaining the lack of observed transiting circumbinary planets in the \kepler field. 

I extended my model for the coupled stellar-tidal evolution of binary stars and applied it to \kepler binaries to probe how stellar evolution, tidal torques, and magnetic braking can shape the rotation period evolution of low-mass binary stars. I showed that my model naturally reproduced the population of short-period subsynchronous Kepler eclipsing binaries discovered by \citet{Lurie2017}. Moreover, I explained how tidal torques can often force the rotation period evolution of stellar binaries to depart from the long-term magnetic braking-driven spin down experienced by single stars revealing that the stellar rotation period is not always a valid proxy for age, i.e. gyrochronology can underestimate ages by up to 300\%.

I combined my models for stellar evolution with Bayesian inference via Markov chain Monte Carlo sampling to put probabilistic constraints on the X-ray and ultraviolet (XUV) emission history of TRAPPIST-1 and understand the evolving high-energy radiation environment experienced by its planets. I inferred that there is a ${\sim}40\%$ chance that TRAPPIST-1 is still in the saturated phase today, suggesting that it has maintained $L_{XUV}/L_{bol} \approx 10^{-3}$ for billions of years. TRAPPIST-1's planetary system therefore likely experienced a persistent and extreme XUV flux environment, potentially driving significant atmospheric erosion and volatile loss. 

Using my models to infer the evolutionary history of stellar and exoplanetary systems is inherently computationally expensive, however, because it requires running a large number of simulations. To enable Bayesian inference at scale with my models, I created an open-source Python machine learning package for efficient approximate Bayesian inference, \approxposterior. I applied this code to the TRAPPIST-1 inference problem to replace running the computationally expensive \vplanet simulations. I demonstrated that \approxposterior sped up this inference by a factor of 980, dramatically reducing the computational cost.  In this dissertation, I combined theories of stellar evolution and tidal torques with Bayesian inference and machine learning to interpret observational data and characterize the long-term evolution of stars and their planets.}
