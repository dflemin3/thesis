In this dissertation, I developed theoretical and probabilistic models to understand and infer the long-term evolution of single and binary stars. Although my modeling efforts focus on the dynamical evolution of such systems, e.g. how and why a binary's orbit evolves over time, I considered what impact stellar and binary evolutionary processes have on planets orbiting the stars. I employed my models to examine different aspects of the lifetime of stellar and planetary systems, including early stellar interactions with a protoplanetary disk and the evolving high-energy radiation environment experienced by terrestrial planets orbiting late M-dwarfs throughout the pre-main sequence. I confronted my model predictions with observational data from NASA missions like \kepler to constrain unobserved model parameters and develop testable hypotheses for unexplained features in the data. Through my novel theories and inference methods, I explore how the past evolution of stars in planetary systems shape what we observe today.

Much of my work focused on developing a theoretical explanation for anomalous stellar and exoplanet populations discovered by \kepler. The recent discoveries of transiting circumbinary planets (CBPs) by \kepler, for example, raise questions for contemporary planet formation models \citep[see][]{Welsh2014}.  Understanding how these planets form requires characterizing their formation environment, the circumbinary protoplanetary disk, and how the disk and binary interact and change as a result.  In young circumbinary systems, the central binary excites gravitational resonances in the surrounding protoplanetary disk that drive evolution in both the binary orbit and in the disk, likely impacting forming and migrating planets.  In Chapter 2, I probed how these resonant interactions impact binary eccentricity and disk structure evolution by running an ensemble of N-body smooth particle hydrodynamics (SPH) simulations of gaseous protoplanetary disks surrounding binaries based on Kepler-38. I ran these large simulations for $10^4$ binary periods over several initial binary eccentricities, disk scale heights, and resolutions.  I demonstrated that nearly circular binaries weakly couple to the disk via a parametric instability and excite disk eccentricity growth, mostly near the inner-edge of the disk where planets form and migrate.  I showed that binaries with eccentric orbits strongly couple to the disk causing eccentricity growth for both the disk and binary orbit. Moreover, disks orbiting binaries with sufficient orbital eccentricity to strongly couple to the disk develop an $m = 1$ spiral wave launched from the 1:3 eccentric outer Lindblad resonance (EOLR) that corresponds to an alignment of gas particle longitude of periastrons. I found that all binaries also underwent semi-major axis decay due to dissipation from the viscous disk. I then considered how disk-binary interactions impacted the long-term binary dynamical evolution and the evolution and dynamical stability of any CBPs subject to this evolution. 

Curiously, \kepler has identified a distinct lack of transiting CBPs orbiting short period binary stars. One proposed explanation for this dearth is dynamical perturbations by a tertiary companion star coupled with tidal friction that shrinks the central binary and perturbs the CBP's orbit, possibly destabilizing it \citep{Munoz2015,Martin2015b,Hamers2016}.  However to date, no theory has been put forward to explain the lack of CBPs around isolated binaries, i.e. those without a tertiary. In Chapter 3, I posited a mechanism that explains the observed lack of CBPs via coupled stellar-tidal evolution of isolated binary stars. I demonstrated how tidal forces between low-mass, short-period binary stars on the pre-main sequence slow the stellar rotations, transferring rotational angular momentum to the orbit as the stars approach the tidally locked state.  This transfer increases the binary orbital period, expanding the region of dynamical instability around the binary, and ultimately destabilizing CBPs that tend to preferentially orbit just beyond the initial dynamical stability limit.  After the stars tidally lock, I found that angular momentum loss due to magnetic braking can significantly shrink the binary orbit, and hence the region of dynamical stability, over time impacting where surviving CBPs are observed relative to the boundary.  I performed simulations over a wide range of parameter space and found that the expansion of the instability region occurs for most plausible initial conditions and that in some cases, the stability semi-major axis doubles from its initial value.  I then examined the dynamical and observable consequences of a CBP falling within the dynamical instability limit by running N-body simulations of circumbinary planetary systems and found that typically at least one planet is ejected from the system.  I applied my theory to the shortest period \kepler binary that possesses a CBP, Kepler-47, and showed via simulation that its existence is consistent with my model.  Under conservative assumptions, I found that coupled stellar-tidal evolution of pre-main sequence binary stars removes at least one close-in CBP in $87\%$ of multi-planet circumbinary systems. I perform several sensitivity tests and found that my mechanism is effective for initial conditions that are generally consistent with observational and theoretical constraints of stellar and binary system orbital parameters. 

To discover transiting exoplanets, \kepler had to constantly observe the stars in its field, thereby creating a massive trove of the brightness modulations of many low-mass, main sequence stars over the course of the telescope's lifetime \citep{Borucki2003,Borucki2010}. By measuring these brightness variations as a function of time, astronomers can infer that some are caused by star spots moving in-and-out of the field of view, allowing astronomers to measure stellar rotation periods \citep[P$_{rot}$, see][]{McQuillan2014}. In Chapter 4, I extended my model for the long-term secular dynamical evolution of binary stars to examine how tides, stellar evolution, and magnetic braking shape the P$_{rot}$ evolution of low-mass stellar binaries in the \kepler field. I explored this evolution up to binary orbital periods (P$_{orb}$) of 100 d and across a wide range tidal dissipation parameters using two common equilibrium tidal models. I found that many binaries with P$_{orb} \lsim 20$ d tidally lock, and most with $P_{orb} \lsim 4$ d tidally lock into synchronous rotation on circularized orbits. At short P$_{orb}$, tidal torques produced a population of fast rotators that single-star only models of magnetic braking fail to produce.  I showed that in many cases, the competition between magnetic braking and tides produced a population of subsynchronous rotators that persisted for Gyrs, even in short P$_{orb}$ binaries, qualitatively reproducing the subsynchronous eclipsing binaries (EBs) discovered by \citet{Lurie2017} in the \kepler field. Both equilibrium tidal models predicted that binaries can tidally-interact out to P$_{orb} \approx 80$ d, while the Constant Phase Lag tidal model predicted that binaries can tidally lock out to P$_{orb} \approx 100$ d. Tidal torques often forced the P$_{rot}$ evolution of stellar binaries to depart from the long-term magnetic braking-driven spin down experienced by single stars, revealing that P$_{rot}$ is not be a valid proxy for age in all cases, i.e. gyrochronology can underpredict ages by up to $300\%$ unless one accounts for binarity. I concluded this Chapter by suggesting how accurate determinations of orbital eccentricties and P$_{rot}$ can be used to discriminate between which equilibrium tidal models best describes tidal interactions in low-mass binary stars.
 
In Chapter 5, I applied machine learning to enable Bayesian inference with computationally-expensive models. I introduced \approxposterior, an open source Python machine learning package for approximate Bayesian inference using Gaussian process (GP) regression. I implemented and generalized both the ``Bayesian Active Learning for Posterior Estimation" (BAPE, \citet{Kandasamy2017}) and ``Adaptive Gaussian process approximation for Bayesian inference with expensive likelihood functions" (AGP, \citet{Wang2018}) approximate Bayesian inference algorithms. For these algorithms, \approxposterior trains a GP to predict the posterior probability estimated by Bayes' Theorem for a set of model parameters by regressing on a small initial subset of forward model runs. The GP is used within an Markov Chain Monte Carlo (MCMC) sampling method, e.g. \emcee, to efficiently obtain an approximation to the posterior probability distribution, dramatically reducing the computational expense relative to obtaining the posterior using the computationally-expensive forward model. Moreover, I detailed how \approxposterior employs an active learning approach to iteratively improve the GP's predictive performance while minimizing the number of calls to the expensive model required to generate the GP's training set as an additional means of reducing the computational cost.  I outlined the steps that comprise \approxposterior's algorithm and explained how to assess convergence with practical examples and a Python script. I concluded by explaining how to use \approxposterior in conjunction with theoretical models as a general method for efficient approximate inference of unobserved model parameters given data.
 
In Chapter 6, I combined my theoretical modeling, machine learning, and Bayesian inference to model the long-term XUV luminosity evolution of TRAPPIST-1. TRAPPIST-1 is a prime target for atmospheric characterization with the James Webb Space Telescope \citep{Morley2017,Lincowski2018,Lustig2019}, and therefore constraining the evolving high-energy radiation environment experienced by its planetary system is critical to modeling their putative atmospheres and interpreting future observations. Using MCMC, I derived probabilistic constraints for TRAPPIST-1's stellar and XUV evolution that accounted for observational uncertainties, degeneracies between model parameters, and empirical data of low-mass stars. I constrained TRAPPIST-1's mass to $m_{\star} = 0.089 \pm{0.001}$ M$_{\odot}$ and found that its early XUV luminosity likely saturated at $\log_{10}(L_{XUV}/L_{bol}) = -3.03^{+0.23}_{-0.12}$. From the posterior distribution, I inferred that there is a ${\sim}40\%$ chance that TRAPPIST-1 is still in the saturated phase today, suggesting that TRAPPIST-1 has maintained high activity and $L_{XUV}/L_{bol} \approx 10^{-3}$ for several Gyrs. TRAPPIST-1's planetary system therefore likely experienced a persistent and extreme XUV flux environment, potentially driving significant atmospheric erosion and volatile loss. The inner planets likely received XUV fluxes ${\sim}10^3 - 10^4\times$ that of the modern Earth during TRAPPIST-1's ${\sim}1$ Gyr-long pre-main sequence phase. I showed how deriving these constraints via MCMC is computationally non-trivial, so scaling my methods to constrain the XUV evolution of a larger number of M dwarfs that harbor terrestrial exoplanets would incur significant computational expenses. I demonstrated that my open-source Python machine learning package, \approxposterior, accurately and efficiently replicates my analysis using 980 times less computational time and 1330 times fewer simulations than MCMC sampling using \emcee. I showed that \approxposterior derives constraints with mean errors on the best fit values and $1\sigma$ uncertainties of $0.61\%$ and $5.5\%$, respectively, relative to \emcee.

Developing a theory within a Bayesian framework allows modelers to appropriately account for data uncertainties and parameter correlations, an essential requirement for the estimation of the past evolution of stellar and planetary systems. Simulating the complex physical effects required to model this evolution while using Bayesian statistics will inevitably incur significant computational expenses, however, I have demonstrated that \approxposterior can enable such efforts. Future research can apply the methodology discussed above and presented in this thesis to provide insight into the histories of stars and their planets using the models implemented in \vplanet in conjunction with \approxposterior. By using a model with numerous physical effects, e.g. models for stellar evolution, water loss, and tidal dissipation, for example, I could build a probabilistic model for the long-term tidal and atmospheric evolution of some exoplanetary system to assess its present-day habitability and understand its past evolution given suitable observational constraints. 

This approach is of course not limited to the theoretical models implemented in \vplanet. For example, as I described in $\S$~\ref{AP:sec:future}, astronomers cannot directly observe the atmospheric chemical abundances that produce spectral absorption features in transmission spectroscopy observations. Instead, astronomers must infer these abundances from spectra using complex radiative transfer inverse modeling that matches model outputs to the data. In $\S$~\ref{AP:sec:future}, I presented results from an in-progress collaboration that demonstrated how \approxposterior can be used with the radiative transfer code SMART \citep{Meadows1996,Crisp1997} to perform atmospheric chemical abundance inference for a realistic model of an Earth-like TRAPPIST-1e (Lustig-Yaeger et al., in prep). Similar to the results discussed in Chapter 6, this inference requires nearly three orders of magnitude less computational resources than modern MCMC samplers like \emcee, demonstrating the promise of applying novel machine learning methods to Bayesian inference problems with computationally-expensive models.

