Moderns astronomical surveys like NASA's \kepler mission have collected a wealth of data in the search for Earth-like exoplanets. This vast quantity of data has enabled novel statistical investigations of the physical processes that shape the observed populations of stars and their planets. Theoretical models are a critical component of such studies and are required to explain how and why planetary systems have evolved over time. By comparing model predictions with observed data and its uncertainties, a process mathematically formalized by Bayesian inference, we can infer the long-term evolution of planetary systems and come to understand what physical processes have shaped their present state.

The recent discoveries of circumbinary planets by \kepler raise questions for contemporary planet formation models.  Understanding how these planets form requires characterizing their formation environment, the circumbinary protoplanetary disk, and how the disk and binary interact and change as a result.  The central binary excites resonances in the surrounding protoplanetary disk that drive evolution in both the binary orbital elements and in the disk.  In Chapter 2, I probe how these resonant interactions impact binary eccentricity and disk structure evolution by running an ensemble of N-body smooth particle hydrodynamics (SPH) simulations of gaseous protoplanetary disks surrounding binaries based on Kepler-38. I ran these large simulations for $10^4$ binary periods over several initial binary eccentricities, disk scale heights, and resolutions.  I demonstrate that nearly circular binaries weakly couple to the disk via a parametric instability and excite disk eccentricity growth, mostly near the inner-edge of the disk where planets form and migrate.  Binaries with eccentric orbits strongly couple to the disk causing eccentricity growth for both the disk and binary orbit. Disks orbiting binaries with sufficient orbital eccentricity to strongly couple to the disk develop an $m = 1$ spiral wave launched from the 1:3 eccentric outer Lindblad resonance (EOLR) that corresponds to an alignment of gas particle longitude of periastrons. All binaries also underwent semi-major axis decay due to dissipation from the viscous disk. I considered how this evolution impacted the long-term binary dynamical evolution and the evolution and dynamical stability of any CBPs subject to this evolution. With this model for the early dynamical evolution of circumbinary systems, I examine the curious lack of observed transiting CBPs in the \kepler field.

The lack of observed CBPs orbiting short period binary stars raises many questions regarding the origin and evolution of these planets.  One proposed explanation for this deficiency is dynamical perturbations by a tertiary companion star coupled with tidal friction that shrinks the central binary and perturbs the circumbinary planet's orbit, possibly destabilizing it \citep{Munoz2015,Martin2015b,Hamers2016}.  However, to date no theory has been put forward to explain the lack of circumbinary planets around isolated binaries, those without a tertiary. In Chapter 2, I outlined a mechanism that explains the observed lack of CBPs via coupled stellar-tidal evolution of isolated binary stars. I demonstrated how tidal forces between low-mass, short-period binary stars on the pre-main sequence slow the stellar rotations, transferring rotational angular momentum to the orbit as the stars approach the tidally locked state.  This transfer increases the binary orbital period, expanding the region of dynamical instability around the binary, and destabilizing CBPs that tend to preferentially orbit just beyond the initial dynamical stability limit.  After the stars tidally lock, I found that angular momentum loss due to magnetic braking can significantly shrink the binary orbit, and hence the region of dynamical stability, over time impacting where surviving CBPs are observed relative to the boundary.  I performed simulations over a wide range of parameter space and found that the expansion of the instability region occurs for most plausible initial conditions and that in some cases, the stability semi-major axis doubles from its initial value.  I then examined the dynamical and observable consequences of a CBP falling within the dynamical instability limit by running N-body simulations of circumbinary planetary systems and found that typically at least one planet is ejected from the system.  I applied our theory to the shortest period \kepler binary that possesses a CBP, Kepler-47, and showed via simulation that its existence is consistent with my model.  Under conservative assumptions, we found that coupled stellar-tidal evolution of pre-main sequence binary stars removes at least one close-in CBP in $87\%$ of multi-planet circumbinary systems. I perform several sensitivity tests and found that my mechanism is effective for initial conditions that are generally consistent with observational and theoretical constraints of stellar and binary system orbital parameters. 

In Chapter 3, I extended my model for the long-term secular dynamical evolution of binary stars to examine how tides, stellar evolution, and magnetic braking shape the P$_{rot}$ evolution of low-mass stellar binaries. I explored this evolution binary orbital periods of 100 d and across a wide range tidal dissipation parameters using two common equilibrium tidal models. I found that many binaries with P$_{orb} \lsim 20$ d tidally lock, and most with $P_{orb} \lsim 4$ d tidally lock into synchronous rotation on circularized orbits. At short P$_{orb}$, tidal torques produced a population of fast rotators that single-star only models of magnetic braking fail to produce.  I showed that in many cases, the competition between magnetic braking and tides produced a population of subsynchronous rotators that persisted for Gyrs, even in short P$_{orb}$ binaries, qualitatively reproducing the subsynchronous eclipsing binaries (EBs) discovered by \citet{Lurie2017} in the \kepler field. Both equilibrium tidal models predicted that binaries can tidally-interact out to P$_{orb} \approx 80$ d, while the Constant Phase Lag tidal model predicted that binaries can tidally lock out to P$_{orb} \approx 100$ d. Tidal torques often forced the P$_{rot}$ evolution of stellar binaries to depart from the long-term magnetic braking-driven spin down experienced by single stars, revealing that P$_{rot}$ is not be a valid proxy for age in all cases, i.e. gyrochronology can underpredict ages by up to $300\%$ unless one accounts for binarity. I concluded this Chapter by suggesting how accurate determinations of orbital eccentricties and P$_{rot}$ can be used to discriminate between which equilibrium tidal models best describes tidal interactions in low-mass binary stars.
 
In Chapter 4, I introduce \approxposterior, an open source Python machine learning package for approximate Bayesian inference using Gaussian process regression. XXX
 
Finally, in Chapter 5, I modeled the long-term XUV luminosity of TRAPPIST-1 to constrain the evolving high-energy radiation environment experienced by its planetary system. Using Markov Chain Monte Carlo (MCMC), I derived probabilistic constraints for TRAPPIST-1's stellar and XUV evolution that accounted for observational uncertainties, degeneracies between model parameters, and empirical data of low-mass stars. I constrained TRAPPIST-1's mass to $m_{\star} = 0.089 \pm{0.001}$ M$_{\odot}$ and found that its early XUV luminosity likely saturated at $\log_{10}(L_{XUV}/L_{bol}) = -3.03^{+0.23}_{-0.12}$. From the posterior distribution, I inferred that there is a ${\sim}40\%$ chance that TRAPPIST-1 is still in the saturated phase today, suggesting that TRAPPIST-1 has maintained high activity and $L_{XUV}/L_{bol} \approx 10^{-3}$ for several Gyrs. TRAPPIST-1's planetary system therefore likely experienced a persistent and extreme XUV flux environment, potentially driving significant atmospheric erosion and volatile loss. The inner planets likely received XUV fluxes ${\sim}10^3 - 10^4\times$ that of the modern Earth during TRAPPIST-1's ${\sim}1$ Gyr-long pre-main sequence phase. I showed how deriving these constraints via MCMC is computationally non-trivial, so scaling my methods to constrain the XUV evolution of a larger number of M dwarfs that harbor terrestrial exoplanets would incur significant computational expenses. I demonstrated that the open-source Python machine learning package \approxposterior accurately and efficiently replicates my analysis using 980 times less computational time and 1330 times fewer simulations than MCMC sampling using \emcee. I showed that \approxposterior derives constraints with mean errors on the best fit values and $1\sigma$ uncertainties of $0.61\%$ and $5.5\%$, respectively, relative to \emcee.

My methodology in Chapter 5 constrained the unobserved, or latent, parameters that describe the long-term XUV evolution of TRAPPIST-1, conditioned on measurements. In principle, this approach can be extended to obtain evolutionary histories of planetary systems in general.  For example, in Figures~\ref{trap:fig:evol} and \ref{trap:fig:fluxes}, I examined the long-term evolution of TRAPPIST-1 and the evolving XUV fluxes received by its planetary system, respectively, with samples drawn from the posterior distribution. Future research can combine those results with additional physical effects, e.g. water loss or tidal dissipation, to build a probabilistic model for the long-term evolution of the planetary system, given a model for the underlying physics, to characterize its present state. In other words I could infer the evolutionary history of a planet or planetary system given suitable observational constraints. While simulating additional physical effects will inevitably increase the computational expense, I have demonstrated that \approxposterior can enable such efforts and provide insight into the histories of stars and their planets.
