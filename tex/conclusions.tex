In Chapter 1, I... XXX
The recent discoveries of circumbinary planets by $\it Kepler$ raise questions for contemporary planet formation
models.  Understanding how these planets form requires characterizing their formation environment, the circumbinary protoplanetary disc,
and how the disc and binary interact and change as a result.  The central binary excites resonances in the surrounding
protoplanetary disc that drive evolution in both the binary orbital elements and in the
disc.  To probe how these interactions impact binary eccentricity and disk structure evolution, N-body smooth particle
hydrodynamics (SPH) simulations of gaseous protoplanetary disks
surrounding binaries based on Kepler 38 were run for $10^4$ binary
periods for several initial binary eccentricities.  We find that
nearly circular binaries weakly couple to the disc via a parametric
instability and excite disc eccentricity growth.  Eccentric binaries strongly couple to the disc
causing eccentricity growth for both the disc and binary.
Disks around sufficiently eccentric binaries that strongly couple to the disc develop an
$m = 1$ spiral wave launched from the 1:3 eccentric outer Lindblad
resonance (EOLR) that corresponds to an alignment of gas
particle longitude of periastrons. All systems display binary semi-major axis decay due to dissipation from the
viscous disc. 


% The observed lack of circumbinary planets orbiting short period binary stars raises many questions regarding the origin and evolution of these planets.  One proposed explanation for this deficiency is dynamical perturbations by a tertiary companion star coupled with tidal friction that shrinks the central binary and perturbs the circumbinary planet's orbit, possibly destabilizing it.  However, to date no theory has been put forward to explain the lack of circumbinary planets around isolated binaries, those without a tertiary.  Here, 

In Chapter 2, I... XXX
We outline a mechanism that explains the observed lack of circumbinary planets (CBPs) via coupled stellar-tidal evolution of isolated binary stars.  Tidal forces between low-mass, short-period binary stars on the pre-main sequence slow the stellar rotations, transferring rotational angular momentum to the orbit as the stars approach the tidally locked state.  This transfer increases the binary orbital period, expanding the region of dynamical instability around the binary, and destabilizing CBPs that tend to preferentially orbit just beyond the initial dynamical stability limit.  After the stars tidally lock, we find that angular momentum loss due to magnetic braking can significantly shrink the binary orbit, and hence the region of dynamical stability, over time impacting where surviving CBPs are observed relative to the boundary.  We perform simulations over a wide range of parameter space and find that the expansion of the instability region occurs for most plausible initial conditions and that in some cases, the stability semi-major axis doubles from its initial value.  We examine the dynamical and observable consequences of a CBP falling within the dynamical instability limit by running N-body simulations of circumbinary planetary systems and find that typically, at least one planet is ejected from the system.  We apply our theory to the shortest period {\it Kepler} binary that possesses a CBP, Kepler-47, and find that its existence is consistent with our model.  Under conservative assumptions, we find that coupled stellar-tidal evolution of pre-main sequence binary stars removes at least one close-in CBP in $87\%$ of multi-planet circumbinary systems.

% We perform several sensitivity tests and find our mechanism is effective for initial conditions that are generally consistent with observational and theoretical constraints of stellar and binary system orbital parameters. 

In Chapter 3, I examine how tides, stellar evolution, and magnetic braking shape the P$_{rot}$ evolution of low-mass stellar binaries up to P$_{orb}$ of 100 d across a wide range tidal dissipation parameters using two common equilibrium tidal models. I find that many binaries with P$_{orb} \lsim 20$ d tidally lock, and most with $P_{orb} \lsim 4$ d tidally lock into synchronous rotation on circularized orbits. At short P$_{orb}$, tidal torques produce a population of fast rotators that single-star only models of magnetic braking fail to produce.  In many cases, I show that the competition between magnetic braking and tides produces a population of subsynchronous rotators that persists for Gyrs, even in short P$_{orb}$ binaries, qualitatively reproducing the subsynchronous eclipsing binaries (EBs) discovered in the \kepler field by \citet{Lurie2017}. Both equilibrium tidal models predict that binaries can tidally-interact out to P$_{orb} \approx 80$ d, while the Constant Phase Lag tidal model predicts that binaries can tidally lock out to P$_{orb} \approx 100$ d. Tidal torques often force the P$_{rot}$ evolution of stellar binaries to depart from the long-term magnetic braking-driven spin down experienced by single stars, revealing that P$_{rot}$ is not be a valid proxy for age in all cases, i.e. gyrochronology can underpredict ages by up to $300\%$ unless one accounts for binarity. I conclude this Chapter by suggesting how accurate determinations of orbital eccentricties and P$_{rot}$ can be used to discriminate between which equilibrium tidal models best describes tidal interactions in low-mass binary stars.
 
In Chapter 4, I introduce \approxposterior, an open source Python machine learning package for approximate Bayesian inference using Gaussian process regression. XXX
 
Finally, in Chapter 5, I model the long-term XUV luminosity of TRAPPIST-1 to constrain the evolving high-energy radiation environment experienced by its planetary system. Using Markov Chain Monte Carlo (MCMC), I derive probabilistic constraints for TRAPPIST-1's stellar and XUV evolution that account for observational uncertainties, degeneracies between model parameters, and empirical data of low-mass stars. I constrain TRAPPIST-1's mass to $m_{\star} = 0.089 \pm{0.001}$ M$_{\odot}$ and find that its early XUV luminosity likely saturated at $\log_{10}(L_{XUV}/L_{bol}) = -3.03^{+0.23}_{-0.12}$. From the posterior distribution, I infer that there is a ${\sim}40\%$ chance that TRAPPIST-1 is still in the saturated phase today, suggesting that TRAPPIST-1 has maintained high activity and $L_{XUV}/L_{bol} \approx 10^{-3}$ for several Gyrs. TRAPPIST-1's planetary system therefore likely experienced a persistent and extreme XUV flux environment, potentially driving significant atmospheric erosion and volatile loss. The inner planets likely received XUV fluxes ${\sim}10^3 - 10^4\times$ that of the modern Earth during TRAPPIST-1's ${\sim}1$ Gyr-long pre-main sequence phase. Deriving these constraints via MCMC is computationally non-trivial, so scaling my methods to constrain the XUV evolution of a larger number of M dwarfs that harbor terrestrial exoplanets would incur significant computational expenses. I demonstrate that \approxposterior accurately and efficiently replicates my analysis using 980 times less computational time and 1330 times fewer simulations than MCMC sampling using \emcee. I find that \approxposterior derives constraints with mean errors on the best fit values and $1\sigma$ uncertainties of $0.61\%$ and $5.5\%$, respectively, relative to \emcee.

